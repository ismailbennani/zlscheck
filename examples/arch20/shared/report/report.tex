\documentclass[a4paper]{easychair}
\usepackage[T1]{fontenc}
\usepackage[utf8]{inputenc}
\usepackage[scaled=0.8]{beramono}

\usepackage{todonotes}
\usepackage{cleveref}
\usepackage{xspace}
\usepackage{booktabs}
\usepackage{multirow}
\usepackage{enumitem}

\usepackage{newunicodechar}
\newunicodechar{□}{\Box}
\newunicodechar{◇}{\Diamond}
\newunicodechar{○}{\circ}

\title{ARCH-COMP 2019 Category Report: Falsification%
       \footnote{The falsification category was coordinated by the first author.
                 The remaining authors represent all participants and they are listed alphabetically.}}
\author{
    Gidon~Ernst\inst{1}
    \and
    Paolo~Arcaini \inst{2} \and
    Alexandre Donze \inst{3} \\
    Georgios~Fainekos \inst{4} \and
    Logan~Mathesen \inst{4} \and 
    Giulia Pedrielli \inst{4} \and \\
    Shakiba~Yaghoubi \inst{4} \and
    Yoriyuki~Yamagata \inst{5} \and
    Zhenya~Zhang \inst{2}
}

\institute{
	Ludwig-Maximilians-University (LMU), Munich, Germany \\
	\email{gidon.ernst@sosy.ifi.lmu.de}
\and
    National Institute of Informatics (NII), Tokyo, Japan \\
    \email{\{arcaini,zhangzy\}@nii.ac.jp}
\and
    Decyphir SAS, Moirans, France \\
    \email{alex@decyphir.com}
\and
    Arizona State University (ASU), Tempe, USA \\
    \email{\{fainekos,lmathese,gpedriel,syaghoub\}@asu.edu}
\and
    National Institute of Advanced Industrial Science and Technology (AIST), Osaka, Japan \\
    \email{yoriyuki.yamagata@aist.go.jp}
}

\titlerunning{ARCH-COMP 2019: Falsification}
\authorrunning{Ernst et al.}

\newcommand{\ARCH}{\url{https://cps-vo.org/group/ARCH/FriendlyCompetition}}
\newcommand{\gitlab}{\url{https://gitlab.com/goranf/ARCH-COMP}}

\newcommand{\STaLiRo}{S-TaLiRo\xspace}
\newcommand{\Breach}{Breach\xspace}
\newcommand{\FalStar}{\textsc{FalStar}\xspace}
\newcommand{\falsify}{falsify\xspace}

\newcommand{\FR}{\ensuremath{\mathit{FR}}}
\newcommand{\Savg}{\ensuremath{\overline{S}}}
\newcommand{\Smed}{\ensuremath{\widetilde{S}}}

\begin{document}
	\maketitle

	\begin{abstract}
      This report presents the results from the 2019 friendly
      competition in the ARCH workshop for the falsification of
      temporal logic specifications over Cyber-Physical Systems.  We
      describe the organization of the competition and how it differs
      from previous years.  We give background on the participating
      teams and tools and discuss the selected benchmarks and results.
      The benchmarks are available on the ARCH
      website\footnote{\ARCH}, as well as in the competition's
      \texttt{gitlab} repository\footnote{\gitlab}.
      The main outcome of the 2019 competition is a common benchmark repository,
      and an initial base-line for falsification, with results from multiple tools,
      which will facilitate comparisons and tracking of the state-of-the-art
      in falsification in the future.
	\end{abstract}

	\section{Introduction}

	The friendly competition of the ARCH workshop is running
    yearly since~2014. The goal is to compare the state-of-the-art of
    tools for testing and verification of hybrid systems.  The
    competition is organized in several categories, with different
    specifications (computing reachable regions, checking temporal
    properties) and varying dynamics in the system models (such as
    linear/non-linear and hybrid).

	In the falsification category, benchmarks typically consist of
    executable Matlab code or Simulink/Stateflow models, each
    associated with a set of requirements in temporal logic with time
    bounds, encoded in MTL~\cite{Koymans1990} or STL~\cite{MalerN04}.
    The task is to find initial conditions and time-varying inputs
    subject to given constraints that steer the system into a
    violation of the respective requirement.  This search is typically
    guided using well-established robustness
    metrics~\cite{FainekosPappas2009} that give a quantitative account
    of how close a given input is to violating a requirement.  Using such
    metrics as score functions permits one to employ standard
    optimization techniques to find falsifying inputs.  Recent results
    in falsification have produced a variety of techniques, mature
    tools, and practical applications, see~\cite{BDDFMNS2018} for an
    overview.  Due to the complexity and unclear semantics of Matlab
    and Simulink models, many previous techniques are entirely
    black-box and just observe the input/output behavior of the system
    via simulations, but grey-box approaches have been developed
    recently~\cite{YaghoubiHSCC,DBLP:conf/fm/AkazakiLYDH18} to take
    some knowledge on the internals of the system into deliberation.
    This year's falsification competition featured more tools and
    participating teams in comparison to previous years, prompting a
    few changes to the organization with respect to benchmark
    solicitation and specification, as well as the validation of the
    results (Sec~\ref{sec:organization}).

    The participating tools~2019 were \STaLiRo~\cite{annpureddy2011s},
    \Breach~\cite{Donze2010},
    \FalStar~\cite{ZhangESAH2018-MCTS,ErnstSZH2018-FalStar}, and
    \falsify~\cite{DBLP:conf/fm/AkazakiLYDH18} in different
    configurations (Sec~\ref{sec:participants}).

	There were 6 benchmark models overall with individual requirements,
    taken from previous competitions and from the literature (Sec~\ref{sec:benchmarks}):
	Automatic Transmission (AT),
	Fuel Control of an Automotive Powertrain (AFC),
	Neural-network Controller (NN),
	Wind Turbine (WT),
	Chasing cars (CC),
	Aircraft Ground Collision Avoidance system (F16),
	and Steam Condenser with Recurrent Neural Network Controller (SC).

    As expected, the results (Sec~\ref{sec:results}) show that tools
    perform better on some benchmarks and worse on others, and that
    different tools have different strengths.  Notable success is
    achieved by the tool \falsify~\cite{DBLP:conf/fm/AkazakiLYDH18} ,
    which often performs best by far, sometimes requiring just a
    single simulation to find a falsifying input thanks to its
    incremental strategy.

	\section{Organization}
	\label{sec:organization}

	Several major changes to the format of the competitions were made in 2019.

	\paragraph{Benchmark Solicitation.}

	For a more comprehensive assessment of the capabilities of tools,
    the benchmark pool was extended to cover several models with
    multiple requirements.  Source of these benchmarks were known
    models from the literature that had served in evaluations of new
    methods and at previous years of the competition.

	Two changes and the presence of more teams than in the years
    before (see \cref{sec:participants}) prompted a more systematic
    organization.  To this end, benchmarks were collected in a fork of
    the \texttt{gitlab} repository in which the reproducibility
    packages are maintained.  Participating teams had access to that
    repository to contribute and download benchmarks models.

	Similarly, descriptions on the use of the models and a precise,
    informal characterization of the respective settings was
    communicated in this way, too.

    \paragraph{Input Specifications.}

	There was some discussion on the format and strictness of the
    input characterizations (for time varying inputs).  Two options
    were proposed and it was agreed to run separate evaluations for
    each of the two options (called instance~1 and instance~2 in the
    following).

	\emph{Arbitrary piece-wise continuous input signals (Instance 1).}
    This option leaves the input specification up to the participants.
    The search space is, in principle, the entire set of piece-wise
    continuous input signals (i.e., which permit discontinuities),
    where the values for each individual
    dimensions are from a given range.  Additional constraints that
    were suggested are finite-number of discontinuity and finite
    variability for all continuous parts of inputs.  Further, each
    benchmark may impose further constraints.  Participants may
    instruct their tools to search a subset of the entire search
    space, notably to achieve finite parametrization, and then to
    apply an interpolation scheme to synthesize the input signal.

	However, the participants agreed that such a choice must be
    ``reasonable'' and should be justified from the problem's
    specification without introducing external knowledge about
    potential solutions.  Moreover, more general parametrizations that
    are shared across requirements and benchmark models were
    preferable.  Due to the diversity of benchmarks, it was decided to
    evaluate the proposed solutions using common sense.

	\emph{Constrained input signals (Instance 2).}  This option
    precisely fixes the format of the input signal, potentially
    allowing discontinuities. An example input signal would be
    piecewise constant with $k$ equally spaced control points, with
    ranges for each dimension of the input, disabling interpolation at
    Simulink input ports so that tools don't need to up-sample their
    inputs.  The arguments in favor of that are increased
    comparability of results.  As possible downside was mentioned that
    optimization-based tools (\STaLiRo and \Breach) are just compared with
    respect to their optimization algorithm.
    Nevertheless such a comparison is still meaningful, in particular,
    as as \FalStar and \falsify implement other approaches to falsification.

    \paragraph{Validation and Repeatability.}

    To prevent implementation bugs and specification errors and to
    check for differences in implementations of robustness
    evaluations, it was planned early on to validate all results.  The
    group's organizer provided a Matlab-based interface for
    validation, implemented by~\FalStar.  Moreover, the organizers of
    the ARCH workshop decided that all groups in the competition
    should submit Docker-based packages for repeatability evaluation,
    but in the end it was infeasible to package an entire Matlab
    installation including the necessary toolboxes.

	\section{Participants}
	\label{sec:participants}

	\paragraph{\STaLiRo.} \STaLiRo~\cite{annpureddy2011s} is a Matlab
    toolbox for monitoring and test case generation against system
    specifications presented in STL. The test cases are automatically
    generated using optimization techniques guided by formal
    requirements in STL in order to find falsifying systems behaviors.
    The tool has different optimization algorithms. Specifically, in
    this competition, the stochastic optimization with adaptive
    restarts (SOAR) \cite{mathesen2019soar} framework is used for all
    the benchmarks except for choosing instance 1 type inputs in Steam
    Condenser model. In that benchmark Simulated annealing global
    search was combined by a local optimal control based search
    \cite{YaghoubiHSCC}.  \STaLiRo is publicly available on-line under
    General Public License (GPL) \footnote{\url{
        https://sites.google.com/a/asu.edu/s-taliro}}.

	\paragraph{\Breach.} \Breach~\cite{Donze2010} is a Matlab toolbox
    for test case generation, formal specification monitoring and
    optimization-based falsification and mining of requirements for
    hybrid dynamical systems. A particular emphasis is put on
    modularity and flexibility of inputs generation, requirement
    evaluation and optimization strategy. For this work, the approach
    has been to ensure that each benchmark was properly implemented
    and a default, relatively basic falsification strategy has been
    applied. The idea was to perform a first systematic investigation
    of the proposed problems, and then to provide a base to work on
    for future editions of the competition to test a larger variety on
    approaches on the most challenging instances.  Breach is available
    under BSD license\footnote{\url{https://github.com/decyphir/breach}}.

	\paragraph{\FalStar.}
	\FalStar is an experimental prototype of a falsification tool that
    explores the idea to construct falsifying inputs incrementally in
    time, thereby exploiting potential time-causal dependencies in the
    problem.  It implements several algorithms, of which two were used
    in the competition: A two layered framework combining Monte-Carlo
    tree search (MCTS) with stochastic
    optimization~\cite{ZhangESAH2018-MCTS}, and a probabilistic
    algorithm~\cite{ErnstSZH2018-FalStar} that adapts to the
    difficulty of the problem dubbed adaptive Las-Vegas tree search
    (aLVTS).  The code is publicly available under the BSD license.%
	\footnote{\url{https://github.com/ERATOMMSD/falstar}}

	\paragraph{\falsify.}
	\falsify is an experimental program which solves falsification
    problems of safety properties by reinforcement
    learning~\cite{DBLP:conf/fm/AkazakiLYDH18}.  \falsify uses a
    \emph{grey-box} method, that is, it learns system behavior by
    observing system outputs during simulation.  \falsify is currently
    implemented by a deep reinforcement learning algorithm
    \emph{Asynchronous Advantage Actor-Critic}
    (A3C)~\cite{DBLP:conf/icml/MnihBMGLHSK16}.

	\section{Benchmarks}
	\label{sec:benchmarks}

	\paragraph{Automatic Transmission (AT).}

	This model of an automatic transmission encompasses a controller
    that selects a gear~1 to~4 depending on two inputs (throttle,
    brake) and the current engine load, rotations per minute~$\omega$,
    and car speed~$v$.  It is a standard falsification benchmark
    derived from a model by Mathworks and has been proposed for
    falsification in~\cite{ARCH14}.

	Input specification: $0 \le \mathit{throttle} \le 100$ and
    $0 \le \mathit{brake} \le 325$ (both can be active at the same
    time).  Constrained input signals (instance~2) permit
    discontinuities at most every 5~time units.

	Requirements (where $○~\phi \equiv ◇_{[0.001, 0.1]}~\phi$) are specific versions of those in \cite{ARCH14} where the parameters have been chose to be somewhat difficult. \\

	\begin{tabular}{ll}
		\toprule
		Benchmark & STL~formula \\
		\cmidrule(r){1-1} \cmidrule(l){2-2}
		AT1  & $□_{[0, 20]} v < 120$ \\
		AT2  & $□_{[0, 10]} \omega < 4750$ \\
		AT51 & $□_{[0, 30]} ((\lnot g1 \land ○~g1) \to ○~□_{[0, 2.5]} g1)$ \\
		AT52 & $□_{[0, 30]} ((\lnot g2 \land ○~g2) \to ○~□_{[0, 2.5]} g2)$ \\
		AT53 & $□_{[0, 30]} ((\lnot g3 \land ○~g3) \to ○~□_{[0, 2.5]} g3)$ \\
		AT54 & $□_{[0, 30]} ((\lnot g4 \land ○~g4) \to ○~□_{[0, 2.5]} g4)$ \\
		AT6a & $(□_{[0, 30]} \omega < 3000) \to (□_{[0,  4]} v < 35)$ \\
		AT6b & $(□_{[0, 30]} \omega < 3000) \to (□_{[0,  8]} v < 50)$ \\
		AT6c & $(□_{[0, 30]} \omega < 3000) \to (□_{[0, 20]} v < 65)$ \\
		\bottomrule
	\end{tabular}

    % \falsify: Input specification for Instance 1 uses piecewise constant function with discontinuities spaced in even intervals $\Delta T$.
	% $\Delta T = 1$ for all models except for SC in which $\Delta T = 0.1$ is used.

	\paragraph{Fuel Control of an Automotive Powertrain (AFC).}

	The model is described in \cite{JinHSCC2014} and has been used in
    two previous instalments of this
    competition~\cite{ARCH17falsification,ARCH18falsification}.  The
    specific limits used in the requirements are chosen such that
    falsification is possible but reasonably hard.

	The constrained input signal (instance~2) fixes the
    throttle~$\theta$ to be piecewise constant with 10 uniform
    segments over a time horizon of 0 with two modes (normal and power
    corresponding to feedback and feedforward control), and the engine
    speed~$\omega$ to be constant with $900 \le \omega < 1100$ to
    capture the input profile outlined in \cite{JinHSCC2014} and to
    match the previous competitions.  For this reason, we do not
    consider the unconstrained (instance~1) input specification.
    Faults are disabled (e.g. by setting $\mathtt{fault\_time} > 50$).

	\begin{tabular}{lll}
		\toprule
		Benchmark & STL~formula & Input Constraint \\
		\cmidrule(r){1-1} \cmidrule(lr){2-2} \cmidrule(l){3-3}
		AFC27 & $□_{[11, 50]} ((\mathit{rise} \lor \mathit{fall}) \to (□_{[1,5]} |\mu| < \beta))$ & $0 \le \theta < 61.2$ (normal) \\
		AFC29 & $□_{[11, 50]} |\mu| < \gamma$ & $0 \le \theta < 61.2$ (normal) \\
		AFC33 & $□_{[11, 50]} |\mu| < \gamma$ & $61.2 \le \theta \ge 81.2$ (power) \\
		\bottomrule
	\end{tabular}

\noindent where
	%
	\begin{align*}
		\beta &= 0.008 \qquad \gamma = 0.007 \\
		\mathit{rise} &= (\theta <  8.8) \land (◇_{[0, 0.05]} (\theta > 40.0))\\
		\mathit{fall} &= (\theta > 40.0) \land (◇_{[0, 0.05]} (\theta <  8.8))
	\end{align*}
	%
	Note that the time interval starts at 11, which is one time unit after the feedback mode is enabled at time~10.

	\paragraph{Neural-network Controller (NN).}

	This benchmark is based on MathWork's neural network controller for a system that levitates a magnet above an electromagnet at a reference position.%
	\footnote{\url{https://au.mathworks.com/help/deeplearning/ug/design-narma-l2-neural-controller-in-simulink.html}}
	It has been used previously as a falsification demonstration in the distribution of Breach.
	The model has one input, a reference value $\mathit{Ref}$ for the position, where $1 \le \mathit{Ref}$ and $\mathit{Ref} \le 3$.
	It outputs the current position of the levitating magnet~$\mathit{Pos}$.
	The input specification for instance~1 requires discontinuities to be at least 3~time units apart,
	whereas instance~2 specifies an input signal with exactly three constant segments.
	The time horizon for the problem is~40.
	The requirement ensures that after changes to the reference, the actual position eventually stabilizes around that value with small error.

\noindent\begin{tabular}{ll}
	\toprule
	Benchmark & STL~formula \\
	\cmidrule(r){1-1} \cmidrule(l){2-2}
	NN & $□_{[1, 37]} \big(|Pos - Ref| > \alpha + \beta |Ref| \to ◇_{[0, 2]} □_{[0, 1]} \lnot (\alpha + \beta |Ref| \le |Pos - Ref|)\big)$ \\
	\bottomrule
\end{tabular}

\noindent where $\alpha=0.005$ and $\beta=0.03$.

    % \falsify: Input specification for Instance 1 uses piecewise constant function with discontinuities spaced in even intervals $\Delta T$.
	% $\Delta T = 1$ for all models except for SC in which $\Delta T = 0.1$ is used.

	\paragraph{Wind Turbine (WT).}

	The model is a simplified wind turbine model proposed
    in~\cite{ARCH16:Hybrid_Modelling_of_Wind}.  The input of the
    system is wind speed $v$ and the outputs are blade pitch angle
    $\theta$, generator torque $M_{g, d}$, rotor speed $\Omega$ and
    demanded blade pitch angle $\theta_d$.  The wind speed is
    constrained by $8.0 \leq v \leq 16.0$.  Instance 1 allows any
    piece-wise continuous inputs, while instance 2 constrains inputs
    to piece-wise constant signals whose control points which are
    evenly spaced each 5 seconds.  The model is relatively large.
    Further, the time horizon is long (630) compared to other
    benchmarks.

	\begin{tabular}{ll}
		\toprule
		Benchmark & STL~formula \\
		\cmidrule(r){1-1} \cmidrule(l){2-2}
		WT1  & $□_{[30, 630]} \theta \leq 14.2$ \\
		WT2  & $□_{[30, 630]} 21000 \leq M_{g, d} \leq 47500$ \\
		WT3 &  $□_{[30, 630]} \Omega \leq 14.3$ \\
		WT4 &  $□_{[30, 630]} ◇_{[0,5]} |\theta - \theta_d| \leq 1.6$ \\
		\bottomrule
	\end{tabular}

    % \falsify: Input specification for Instance 1 uses piecewise constant function with discontinuities spaced in even intervals $\Delta T$.
    % $\Delta T = 1$ for all models except for SC in which $\Delta T = 0.1$ is used.

	\paragraph{Chasing cars (CC).}

	The model is derived from Hu et al.~\cite{hu2000towards} which
    presents a simple model of an automatic chasing car.  Chasing cars
    (CC) model consists of five cars, in which the first car is driven
    by inputs ($\mathit{throttle}$ and $\mathit{brake}$), and other
    four are driven by Hu et al.'s algorithm.  The output of the
    system is the location of five cars $y_1, y_2, y_3, y_4, y_5$.
    The properties to be falsified are constructed artificially, to
    investigate the impact of complexity of the formulas to
    falsification.  The input specifications for instance 1 allows any
    piecewise continuous signals while the input specification for
    instance 2 constraints inputs to piecewise constant signals with
    control points for each 5 seconds, i.e., 20 segments

	\begin{tabular}{ll}
		\toprule
		Benchmark & STL~formula \\
		\cmidrule(r){1-1} \cmidrule(l){2-2}
		CC1  & $□_{[0, 100]} y_5 - y_4 \leq 40$ \\
		CC2  & $□_{[0, 70]} ◇_{[0,30]} y_5 - y_4 \geq 15$ \\
		CC3 &  $□_{[0, 80]} ((□_{[0,20]}y_2 - y_1 \leq 20) \lor (◇_{[0,20]} y_5 - y_4 \geq 40))$ \\
		CC4 &  $□_{[0, 65]} ◇_{[0,30]} □_{[0,20]} y_5 - y_4 \geq 8$ \\
		CC5 &  $□_{[0, 72]} ◇_{[0,8]} ((□_{[0,5]} y_2 - y_1 \geq 9) \to (□_{[5, 20]} y_5 - y_4 \geq 9))$\\
		\bottomrule
	\end{tabular}

%   \falsify: Input specification for Instance 1 uses piecewise constant function with discontinuities spaced in even intervals $\Delta T$.
%   $\Delta T = 1$ for all models except for SC in which $\Delta T = 0.1$ is used.


	\paragraph{Aircraft Ground Collision Avoidance system (F16).}
	The model has been derived
    from~\cite{ARCH18:Verification_Challenges_in_F_16}. The F16
    aircraft and its inner-loop controller for Ground Collision
    avoidance have been modeled using 16 continuous variables with
    piece-wise nonlinear differential equations. Autonomous maneuvers
    are performed in an outer-loop controller that uses a finite-state
    machine with guards involving the continuous variables. The system
    is required to always avoid hitting the ground during its maneuver
    starting from all the initial conditions for roll, pitch, and yaw
    in the range
    $[0.2\pi, 0.2833\pi]\times[ -0.5\pi, -0.54\pi]\times[0.25\pi,
    0.375\pi]$.  Since the benchmark has no time-varying input, there
    is no distinction between instance~1 and instance~2.  The
    requirement is checked for a time horizon equal to 15. This
    requirement can be captured using the following formula

	\begin{tabular}{ll}
		\toprule
		Benchmark & STL~formula \\
		\cmidrule(r){1-1} \cmidrule(l){2-2}
		F16  & $□_{[0, 15]} altitude >0$ \\
		\bottomrule
	\end{tabular}

	\paragraph{Steam condenser with Recurrent Neural Network Controller (SC).}

	The model is presented in \cite{YaghoubiHSCC}. It is a dynamic model of an steam condenser based on energy balance and cooling water mass balance controlled with a Recurrent Neural network in feedback. The time horizon for the problem is 35 seconds. The input to the system can vary in the range $[3.99, 4.01]$.
    For instance~2, the input signal should be piecewise constant with 20~evenly spaced segments.
    The pressure output should satisfy the following requirement:

	\begin{tabular}{ll}
		\toprule
		Benchmark & STL~formula \\
		\cmidrule(r){1-1} \cmidrule(l){2-2}
		SC  & $□_{[30,35]} (87\leq \mathit{pressure} \land \mathit{pressure}\leq 87.5)
		$ \\
		\bottomrule
	\end{tabular}

    % \falsify: We choose $\Delta T = 0.1$ for SC model because Instance 2 uses $\Delta T = 1.75$, which is near to $\Delta T = 1$.

	\section{Evaluation}
    \label{sec:results}

	Falsification tools were instructed to run each individual
    requirement 50 times, to account for the stochastic nature of most
    algorithms.  We report the falsification rate, i.e., the number of
    trials where a falsifying input was found, as well as the median
    and mean of the number of simulations required to find such input
    (not including the unsuccessful runs in the aggregate).  The
    cut-off for the number of simulations per trial was
    300. % and in case no falsification was achieved, we indicate a table entry ``N/A''.

    The results for unconstrained piecewise-continuous input signals (instance~1)
    are shown in Table~\ref{table:allResultsInstance1}.
    They depend on the choices for the search space,
    which we briefly discuss for each participating tool:

    \paragraph{\Breach.} For most benchmarks (exceptions detailed below), a
    piecewise constant signal generation was used with fixed step
    size. For all instances, the optimization strategy used is the
    default global Nelder Mead (GNM) approach with a custom
    configuration for the competition, resulting in the following
    three phases behavior:
    \begin{itemize}[noitemsep,topsep=0pt,parsep=0pt,partopsep=0pt]
    \item Phase 1.: at most $n_{\text{corners}}= $ 64 corner samples are tested, i.e., inputs
      for which control points take only extreme values;
    \item Phase 2.: $n_{\text{quasi-rand}} = 100-n_{\text{corners}}$
      quasi-random samples from the Halton sequence with
      varying start points determined by a random seed are tested;
    \item Phase 3.: the robustness results from phase 1 and 2 are
      sorted and Nelder Mead optimization is run from the most
      promising samples.
    \end{itemize}
    Note that as a result of this approach, whenever a falsifying
    input is consistently found with less than 100 simulations, it
    indicates that the problem is likely trivially falsifiable with
    extreme inputs or a quick stochastic exploration of the search
    space. The following settings were chosen for input generation
    for each benchmark:
    \begin{itemize}[noitemsep,topsep=0pt,parsep=0pt,partopsep=0pt]
    \item AT: throttle input and brake inputs were configured
      with respectively 3 and 2 control points at variable times;
    \item NN: input was piecewise constant with 3 control
      points regularly spaced;
    \item WT: spline interpolation with control
      points regurlary spaced by 5s and saturation to domain [8;16]
      (same for instance 2);
    \item CC: same as instance 2, i.e., piecewise constant input with control
      points regurlary spaced by 5s;
    \item SC: same as instance 2, i.e., piecewise constant input with control
      points regurlary spaced by 1.75s;
    \end{itemize}


    \paragraph{\STaLiRo.} In S-TaLiRo, input signals are parameterized in two ways:
    the number of control points for the input signal, and the
    time location of those control points during simulation. The number of
    control points for each input signal is given by the user forming
    an optimization problem with search space dimension the same as the number of 
    control points. An option is provided to the user to add to the search space 
    the timing of the control points, but this option is not used in the competition.
    For this competition, the control point time locations are evenly spaced over the duration 
    of the simulation for all the benchmarks except for the SC problem instance 1.

    For the transmission model the $[throttle, brake]$ control points
    are interpolated with the \textit{pchip} function, with $[7, 3]$ as
    the number of control points in specifications 1-6 and $[4,2]$ for
    7-9 to reduce the dimensionality of the search space. For the Neural model, we use $13$
    control points to yield piecewise constant signals of $3.33$ seconds apart.
    %The Chasing Cars model used $5$ control points for piecewise constant signals of $20$ seconds each.
    The Wind Turbine used the default model input of $126$ control points interpolated linearly.
    For the SC model, Simulated Annealing (SA) global search was utilized in combination with an optimal control based local search on the infinite dimensional input space. The SA global search utilizes piecewise constant inputs with 12 possibly uneven time durations.

    \paragraph{\FalStar~(MCTS).}
    The search space included piecewise constant inputs. For all the benchmarks and for all the specifications, the number of control points was computed according to the simulation time that was divided by a fixed interval of 5 time units (e.g., a simulation time of 50 has 10 control points).

    \paragraph{\FalStar~(aLVTS).} The search space included piecewise constant
    inputs (the only parameterization currently supported), ranging
    from~2 upto~4 control points at which discontinuities are allowed
    (resp. upto~3 for NN).  In this configuration \FalStar benefits
    from a low number of control points and is more likely to try
    inputs with fewer control points first.  For the AT benchmarks it
    was clear beforehand that this choice suffices to falsify all
    benchmarks, and the setting was then kept for the remaining
    experiments.

    \paragraph{\falsify.}
	The input specification uses piecewise constant function with discontinuities spaced in even intervals $\Delta T$.
	$\Delta T = 1$ for all models except for SC in which $\Delta T = 0.1$ is used.
	The choice for the SC model was $\Delta T = 0.1$ model because Instance 2 uses $\Delta T = 1.75$, which is near to $\Delta T = 1$.

    \paragraph{Common Settings.}

    For a better comparison of the performance of the tools, a common
    ground is piecewise constant input signals (instance~2) with a
    concrete specification of the number of discontinuities allowed.
    The corresponding results are shown in Table~\ref{table:allResultsInstance2}.

    The implied intention was that the search space is actually large
    enough so that in principle input signals are generated that
    exhibit that number of variability.  For this reason, for example,
    \FalStar~(aLVTS) was instructed not to generate input signals with
    fewer control points (even though it would likely have improved
    the results). \FalStar~(MCTS) similarly takes the number of
    control points as input. % and, therefore, applying instance~2 has not been a problem.

    \begin{table}
	\centering
	\caption{Results for piecewise continuous input signals (instance 1).
		\FR: falsification rate wrt.~number of trials,
		\Savg\ and \Smed: mean resp. median number of simulations over successful trials (``--'' if \FR\ is zero).
	}
	\label{table:allResultsInstance1}
	
	\resizebox{\textwidth}{!}{\begin{tabular}{lrrrrrrrrrrrrrrr}
			\toprule
			\bf
			Tool:
			& \multicolumn{3}{c}{\bf\STaLiRo}
			& \multicolumn{3}{c}{\bf\Breach}
			& \multicolumn{6}{c}{\bf\FalStar}
			& \multicolumn{3}{c}{\bf\falsify}\\
			% \cmidrule(r){2-4}
			% \cmidrule(lr){5-7}
			% \cmidrule(lr){8-13}
			% \cmidrule(l){14-19}
			Configuration:
			& \multicolumn{3}{c}{SOAR}
			& \multicolumn{3}{c}{GNM}
			& \multicolumn{3}{c}{MCTS}
			& \multicolumn{3}{c}{aLVTS}
			& \multicolumn{3}{c}{A3C}\\
			\midrule
			\midrule
			Benchmark
			& \FR & \Savg & \Smed
			& \FR & \Savg & \Smed
			& \FR & \Savg & \Smed
			& \FR & \Savg & \Smed
			& \FR & \Savg & \Smed\\
			\cmidrule(r){1-1} \cmidrule(lr){2-4} \cmidrule(lr){5-7} \cmidrule(lr){8-10} \cmidrule(lr){11-13} \cmidrule(lr){14-16}
			AT1  & 50/50 & 118.84 & 116 &50/50  & 11    & 11  &  0/50 &     -- &  -- & 50/50 &  33.04 &  30 & 17/50 & 224.53 & 223.0\\
			AT2  & 50/50 &23.88   &19.5   &50/50  & 2     & 2   & 50/50 &  21.68 &   5 & 50/50 &   4.32 &   3 & 50/50 & 22.10 & 10.0\\
			AT51 & 50/50 & 26.72  &22    &41/50  & 74.56 & 67  & 50/50 &  39.10 &  43 & 50/50 &  69.48 &  56 & 50/50 & 1.00 & 1.0\\
			AT52 & 50/50 &4.06    &3       &49/50  & 72    & 67  & 50/50 &   2.68 &   2 & 26/50 & 125.35 & 137 & 50/50 & 1.00 & 1.0\\
			AT53 & 50/50 &3.44    &3       &49/50  & 74.51 & 73  & 50/50 &   2.52 &   3 & 50/50 &  70.82 &  68 & 50/50 & 1.00 & 1.0\\
			AT54 & 50/50 &10.46   &2      &21/50  & 84.86 & 85  & 50/50 &  26.56 &  10 & 50/50 &  71.10 &  52 & 50/50 & 1.00 & 1.0\\
			AT6a & 49/50 &78.41   &40     &50/50  & 97.92 & 97.5& 49/50 &  93.65 &  80 & 50/50 &  76.06 &  70 &  \multicolumn{3}{c}{(not a safety property)}  \\
			AT6b & 33/50 &132.64 &128   &49/50  & 112.9 & 118 & 29/50 & 166.90 & 173 & 50/50 &  82.38 &  75 &  \multicolumn{3}{c}{(not a safety property)}  \\
			AT6c &47/50 &61.34   &38      &  50/50  & 94.1  & 89  &6/50 & 105.92 & 125 &  0/50 &     -- &  -- &  \multicolumn{3}{c}{(not a safety property)}\\
			
			\cmidrule(r){1-1} \cmidrule(lr){2-4} \cmidrule(lr){5-7} \cmidrule(lr){8-10} \cmidrule(lr){11-13} \cmidrule(lr){14-16}
			% NOTE (Gidon) no instance 1 benchmark for the powertrain model
			% AFC27 &&&&&&& 50/50 & 4 & 3 &&&&&&&&&\\
			% AFC29 &&&&&&& 50/50 & 2 & 2 &&&&&&&&&\\
			% AFC33 &&&&&&& 50/50 & 2 & 2 &&&&&&&&&\\
			% \cmidrule(r){1-1} \cmidrule(lr){2-4} \cmidrule(lr){5-7} \cmidrule(lr){8-10} \cmidrule(lr){11-13} \cmidrule(lr){14-16} \cmidrule(l){17-19}
			NN  &50/50&26.72&22& 48/50 & 96.29 & 101.5   & 50/50 & 47.98 & 40 & 36/50 & 122.80 & 106 &  50/50 & 1.00 & 1.0\\
			
			NN($\beta$ = 0.04)  &4/50&193&222.5&  &  &   &  &  &  &  & & & & &\\			\cmidrule(r){1-1} \cmidrule(lr){2-4} \cmidrule(lr){5-7} \cmidrule(lr){8-10} \cmidrule(lr){11-13} \cmidrule(lr){14-16}
			WT1 &50/50 &91 &91 &50/50 & 3 & 3& 50/50 & 4.00 & 4 & \multicolumn{3}{c}{(unsupported)} &  37/50 & 47.68 & 7.0\\
			WT2 &50/50 &32.62 & 30 &50/50 & 3 & 3& 50/50 & 1.00 & 1 & & & & 46/50 & 8.04 & 2.0\\
			WT3 &50/50 &44.12& 60 &50/50 & 3 & 3& 50/50 & 2.00 & 2 & & & & 50/50 & 2.48 & 1.0\\
			WT4 &50/50 &3.32 &2&50/50 & 30 &30 & 50/50 & 2.00 & 2 & & & & 50/50 & 4.90 & 4.0\\
			\cmidrule(r){1-1} \cmidrule(lr){2-4} \cmidrule(lr){5-7} \cmidrule(lr){8-10} \cmidrule(lr){11-13} \cmidrule(lr){14-16}
			CC1 &50/50 & 9.38  & 7 & 50/50 &  3    & 3 & 50/50 &  15.00 &  15 & 50/50 &  4.06 &  2 & 47/50 & 51.26 & 17.0 \\
			CC2 &50/50 & 6     & 4 & 50/50 &  1    & 1 & 50/50 &  26.00 &  26 & 50/50 &  4.02 &  2 & 37/50 & 24.24 & 4.0\\
			CC3 &50/50 & 19.94 & 5 & 50/50 &  3    & 3 & 50/50 &  14.40 &  17 & 50/50 &  6.86 &  5 & 46/50 & 35.41 & 8.5\\
			CC4 & 20/50   & 188   & 179.5 &  0/50  &   --  & --&  0/50 &     -- &  -- &  2/50 & 52.00 & 60 &  1/50 & 26.00 & 26.0\\
			CC5 &50/50 & 42.94& 36.5 & 49/50 & 26.08 & 19& 50/50 & 132.00 & 140 & 46/50 & 91.19 & 79 & 31/50 & 29.65 & 26.0\\
			%CC6 &&&&&&& N/A\\
			\cmidrule(r){1-1} \cmidrule(lr){2-4} \cmidrule(lr){5-7} \cmidrule(lr){8-10} \cmidrule(lr){11-13} \cmidrule(lr){14-16}
			F16 &7/50&127.57&94 & 1/50 & 297 & 297 & \multicolumn{3}{c}{(unsupported)} & 0/50\rlap{$\dagger$} & -- & -- & \multicolumn{3}{c}{(unsupported)}\\
			\cmidrule(r){1-1} \cmidrule(lr){2-4} \cmidrule(lr){5-7} \cmidrule(lr){8-10} \cmidrule(lr){11-13} \cmidrule(lr){14-16}
			SC  &50/50  \rlap{$\star$} & 62.26 & 55.5 & 0/50 & -- & -- & 0/50 & -- & -- & 0/50 & -- & -- & 0/50 & -- &--\\
			\bottomrule
	\end{tabular}}
	\begin{flushleft}
		\small F16 $\dagger$: FalStar/aLVTS currently samples initial conditions uniformly at random\\
		\small SC $\star$: The S-TaLiRo results for this benchmark are yielded by Simulated annealing assisted with gradient based search (See \cite{YaghoubiHSCC}).   
	\end{flushleft}
\end{table}

\begin{table}
	\centering
	\caption{Results for constrained input signals/instance 2. FR: falsification rate, $\overline{S}$: mean number of simulations, $\widetilde{S}$: median number of simulations.}
	\label{table:allResultsInstance2}
	\resizebox{\textwidth}{!}{\begin{tabular}{lrrrrrrrrrrrrrrr}
			\toprule
			\bf
			Tool:
			& \multicolumn{3}{c}{\bf\STaLiRo}
			& \multicolumn{3}{c}{\bf\Breach}
			& \multicolumn{6}{c}{\bf\FalStar}
			& \multicolumn{3}{c}{\bf\falsify}\\
			% \cmidrule(r){2-4}
			% \cmidrule(lr){5-7}
			% \cmidrule(lr){8-13}
			% \cmidrule(l){14-19}
			Configuration:
			& \multicolumn{3}{c}{SOAR}
			& \multicolumn{3}{c}{GNM}
			& \multicolumn{3}{c}{MCTS}
			& \multicolumn{3}{c}{aLVTS}
			& \multicolumn{3}{c}{A3C} \\
			\midrule
			\midrule
			Benchmark
			& \FR & \Savg & \Smed
			& \FR & \Savg & \Smed
			& \FR & \Savg & \Smed
			& \FR & \Savg & \Smed
			& \FR & \Savg & \Smed\\
			\cmidrule(r){1-1} \cmidrule(lr){2-4} \cmidrule(lr){5-7} \cmidrule(lr){8-10} \cmidrule(lr){11-13} \cmidrule(lr){14-16}
			AT1  &50/50 & 170.32 & 171  & 0/50   & -- & -- &  0/50 &     -- &  -- & 50/50 &  33.04 &  30 & 39/50 & 125.79 & 110\\
			AT2  &50/50 & 16.82  &9     & 50/50  & 2  & 2  & 50/50 &  21.68 &   5 & 50/50 &   4.32 &   3 & 48/50 & 18.96 & 7.5\\
			AT51 &50/50 &12.64   &11    & 50/50 & 7  & 7 & 50/50 &  39.10 &  43 & 50/50 &  69.48 &  56 & 50/50 & 1.00 & 1.0\\
			AT52 &49/50 &17.57   &4     & 50/50 & 3  & 3 & 50/50 &   2.68 &   2 & 26/50 & 125.35 & 137 & 50/50 & 1.00 & 1.0\\
			AT53 &50/50 & 3.38& 3       & 50/50 & 3  & 3 & 50/50 &   2.52 &   3 & 50/50 &  70.82 &  68 & 50/50 & 1.00 & 1.0\\
			AT54 &50/50 & 24.15& 16.5   & 50/50 & 3  & 3 & 50/50 &  26.56 &  10 & 50/50 &  71.10 &  52 & 50/50 & 1.00 & 1.0\\
			AT6a &44/50 & 130.353 &149.5& 0/50  & -- & --& 49/50 &  93.65 &  80 & 50/50 &  76.06 &  70 & \multicolumn{3}{c}{(not safety)}\\
			AT6b &39/50 & 207.206& 236  & 0/50  & -- & --& 29/50 & 166.90 & 173 & 50/50 &  82.38 &  75 & \multicolumn{3}{c}{(not safety)}\\
			AT6c &42/50 & 197.5& 208.5  & 0/50  & -- & --& 36/50 & 105.92 & 125 &  0/50 &     -- &  -- & \multicolumn{3}{c}{(not safety)}\\
			\cmidrule(r){1-1} \cmidrule(lr){2-4} \cmidrule(lr){5-7} \cmidrule(lr){8-10} \cmidrule(lr){11-13} \cmidrule(lr){14-16}
			AFC27 & 50/50 & 70.34 & 78.5&50/50 &3&3& -- & -- & -- & 50/50 & 3.88 & 3 & 50/50 & 1.64 & 1.0 \\
			AFC29 & 50/50 & 13.54 & 10  &50/50 &3&3 & -- & -- & -- & 50/50 & 1.18 & 1 & 50/50 & 1.00 & 1.0\\
			AFC33 & 0/50 &--&--         & 0/50  & -- &-- & -- & -- & -- &  0/50 &   -- & -- & 50/50 & 1.00 & 1.0\\
			\cmidrule(r){1-1} \cmidrule(lr){2-4} \cmidrule(lr){5-7} \cmidrule(lr){8-10} \cmidrule(lr){11-13} \cmidrule(lr){14-16}
			NN & 49/50 & 67.96 & 48        &50/50&6&6& 50/50 & 177.44 & 183 & 26/50 & 176.96 & 197 & 50/50 & 1.00 & 1.0 \\
			NN($\beta$=0.04) & 3/50 & 127 & 74        &&&& & && & & & & &\\
			\cmidrule(r){1-1} \cmidrule(lr){2-4} \cmidrule(lr){5-7} \cmidrule(lr){8-10} \cmidrule(lr){11-13} \cmidrule(lr){14-16}
			WT1 & 50/50 & 7.1 & 5       &50/50 & 3 & 3& 50/50 & 4 & 4 & \multicolumn{3}{c}{(unsupported)} & 49/50 & 8.57  & 2.0 \\
			WT2 & 50/50 & 1.02 & 1      &50/50 & 3 & 3& 50/50 & 1 & 1 & & & & 50/50 & 2.76 & 2.0 \\
			WT3 &50/50 &1.04 &1         &50/50 & 3 & 3& 50/50 & 2 & 2 & & & & 50/50 & 2.04 & 1.0 \\
			WT4 &50/50 &12.02 &9        &50/50 & 30 & 30& 50/50 & 2 & 2 & & & & 50/50 & 3.68 & 3.0 \\
			\cmidrule(r){1-1} \cmidrule(lr){2-4} \cmidrule(lr){5-7} \cmidrule(lr){8-10} \cmidrule(lr){11-13} \cmidrule(lr){14-16}
			CC1 & 50/50 &67.8 &91       & 50/50 &  5    & 5 & 50/50 &  15.00 &  15 & 50/50 &   7.30 &   6 & 50/50 &  23.48 &   7.0 \\
			CC2  & 48/50 &114.48 &105          & 50/50 &  1    & 1 & 50/50 &  26.00 &  26 & 50/50 &  15.88 &   9 & 46/50 &  14.41 &   4.0 \\
			CC3  & 50/50 & 22.36 & 13   & 50/50 &  5    & 5 & 50/50 &  14.40 &  17 &  4/50 & 207.50 & 229 & 44/50 &  13.52 &   2.5 \\
			CC4 & 0/50 & --& --           & 0/50  &  --  & --&  0/50 &     -- &  -- &  0/50 &    --  &  -- &  9/50 & 120.44 & 168.0 \\
			CC5 & 50/50 & 74.88& 48     & 16/50 & 84.69 & 79& 50/50 & 132.00 & 140 & 39/50 & 117.94 & 103 & 32/50 &  37.22 &   8.5 \\
			%CC6 &&&&&&& N/A\\
			% \cmidrule(r){1-1} \cmidrule(lr){2-4} \cmidrule(lr){5-7} \cmidrule(lr){8-10} \cmidrule(lr){11-13} \cmidrule(lr){14-16}
			% F16 &&&&&&& \multicolumn{3}{c}{(unsupported)} & \multicolumn{3}{c}{(missing toolbox)} & \multicolumn{3}{c}{(parameter generation problem)}\\
			\cmidrule(r){1-1} \cmidrule(lr){2-4} \cmidrule(lr){5-7} \cmidrule(lr){8-10} \cmidrule(lr){11-13} \cmidrule(lr){14-16}
			SC &0/50& -- & -- & 0/50 & -- & -- & 0/50 & -- & -- & 0/50 & -- & -- & 0/50 & -- & -- \\
			\bottomrule
	\end{tabular}}
\end{table}

%%% Local Variables:
%%% mode: latex
%%% TeX-master: "report"
%%% End:


    \paragraph{Discussion.}

    The most successful outcome can be attributed to \falsify, which
    can find a falsifying input signal with a single simulation only.
    This is remarkable as all sampling and tuning of the input is done
    on the fly with respect to an established prefix.  Notably, most
    results for AT, AFC, NN, and WT models show this behavior.  While
    the strategy works well enough and applies generally, there are a
    few instances where other strategies work better, e.g., the
    tendency of GNM and aLVTS to prefer extreme values for AT1, as
    well as the \STaLiRo's algorithm used to solve the SC benchmark.

    Another observation is that many benchmarks can be solved quite
    easily, leading to perfect falsification rates (FR) of 50/50.
    This suggests that the overall mass of falsifying inputs is large
    enough so that those can be found reliably with significantly less
    than 300 simulations, independently of whether sophisticated
    algorithms are used (\STaLiRo, MCTS, \falsify), or whether
    sampling is random (but not necessarily uniform, GNM, aLVTS).
    This can be seen as a confirmation of folklore knowledge that many
    falsification problems are actually not that hard to solve, and
    that conceptually simple methods can indeed work well.  A clear
    take-away is that the set of benchmarks needs to be more
    challenging in the future.

    There is not so much of a difference between the results for
    instance~1 and instance~2.  The latter typically require more time
    to solve, as the parameter space is larger (e.g., the~CC benchmark
    has many more control points in the second instance).

    Due to the inhomogeneity of results we refrain from a further
    in-depth analysis of when and why exactly certain algorithms fare
    better than others.  This is left for future work (cf.~also the
    comment on benchmark classification below).  Similarly, we do not
    attempt to derive an aggregate score to determine a ``winner''.

    \section{Conclusion and Outlook}

    Setting up the competition with the format described in
    Sec.~\ref{sec:organization} proved to be difficult and
    time-consuming.

    The first issue is a general lack of a well-prepared benchmark
    suite for falsification of black-box systems.  In comparison to
    the other tracks (and similar competitions such as SMT-COMP and
    SV-COMP), the ARCH group on falsification could previously not
    refer to an established repository of benchmarks.  By collecting
    existing benchmarks and soliciting new ones in a single place now
    we have that for the first time.

    Nevertheless, even though all participating tools interface with
    Matlab/Simulink, there are differences in how this is done, e.g.,
    how input parameters are passed to the simulation engine.  As an
    example, \Breach and \STaLiRo internally interpolate input signals
    to a high sample rate, whereas \FalStar relies on Simulink for
    this and provides compressed input signals with a low sample rate.
    As a consequence, the result for \FalStar differs depending on the
    interpolation setting of the input ports in the model, whereas
    that is not the case for the other tools.  Another issue that had
    to be clarified was the input signal specification, as discussed
    in Sec.~\ref{sec:organization}.  Repeatability and
    cross-validation of results currently involves a lot of manual
    set-up due to lack of common standards for reporting and
    processing these.  For future competitions, it would be much
    better, if such standards are established beforehand, so that the
    initial phase of the competition can focus on benchmark selection.
    In the end, the envisioned cross-checking of results happened not
    systematically, so the results cannot be trusted entirely, but at
    least a partial success in this regard could be achieved.

    Running the benchmarks is computationally expensive, which is
    aggravated by the need to run many trials (50 here) to account for
    the stochastic nature of the algorithms.  As a consequence,
    running all benchmarks for a given tools takes several machine-days.
    In combination with the discussions, this factor lead to a delayed
    availability of stable results, and a short time frame for
    validating the results.  It would be great to use existing
    computing infrastructure such as StarExec%
    \footnote{\url{https://www.starexec.org/starexec/public/about.jsp}}
    to run the benchmarks, but it is currently unclear whether this is
    feasible with respect to availability of sufficient Matlab
    licenses.

    We hope that future competition installments will benefit from the
    now established set up, and that furthermore this benchmark
    repository will eventually serve as a base-line for experimental
    research in the falsification community.  Moreover, with respect
    to such a base-line it will be easier to track the
    state-of-the-art and progress of the effectiveness and efficiency
    of falsification tools.  Potential future work would be to include
    a classification of the difficulties of the individual benchmarks
    (and the falsification rate of uniform random sampling), so that
    it can be investigated which approach works well for which kind of
    problem.

\paragraph{Acknowledgments}
P. Arcaini and Z. Zhang are supported by ERATO HASUO Metamathematics for Systems Design Project (No. JPMJER1603), JST. Funding Reference number: 10.13039/501100009024 ERATO.
The report on \falsify is based on results obtained from a project commissioned by the New Energy and Industrial Technology Development Organization (NEDO).
The ASU team (\STaLiRo) was partially supported by NSF CNS 1350420, NSF CMMI 1829238, NSF IIP-1361926 and the NSF I/UCRC Center for Embedded Systems.


	\bibliographystyle{plain}
	\bibliography{references}

\end{document}
