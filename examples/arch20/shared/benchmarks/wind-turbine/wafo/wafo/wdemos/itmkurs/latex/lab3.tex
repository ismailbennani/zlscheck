\renewcommand{\FileId}{File: lab3.tex, Last changed: 2005-04-14}

%%%%%%%%%%%%%%%%%%%%%%%%%%%%%%%%%%%%%%%%%%%%%%%
% Power Spectrum and Rainflow Analysis
%%%%%%%%%%%%%%%%%%%%%%%%%%%%%%%%%%%%%%%%%%%%%%%

%\cleardoublepage
\chapter{Power Spectrum and Rainflow Analysis}
\label{lab3}

In this exercise we present tools for computation of
rainflow matrices (and consequently fatigue failure predictors)
for Gaussian random loads. As was mentioned in Introduction,
a Gaussian load is uniquely defined by its mean value $m$ and
spectrum $S(\omega)$. For simplicity only, the mean is assumed to be
zero;  hence, the spectrum is the single input argument into the model.
The spectrum can be estimated from a measured record
(as we did in Computer Exercise 1) or theoretically derived,
as it is often the case  in fatigue damage analysis of sea vessels.

\section{Power Spectrum}

Choose any spectrum you wish to analyse. It can be an estimated
spectrum from Lab 1 or some standard spectra given in WAFO. Four
examples are given below:
\begin{code}
>> S=jonswap;
>> S=oscspec;
>> S=torsethaugen;
>> load deep.dat, S=dat2spec(deep);
\end{code}
Compute mean frequency $f_0$, standard deviation $\sigma$, and
irregularity factor $\alpha$ for the Gaussian process with computed
spectrum $S$. Solution:
\begin{code}
>> L=spec2mom(S,4);
>> f0=sqrt(L(2)/L(1))/2/pi
>> s=sqrt(L(1))
>> alfa=f0/(sqrt(L(3)/L(2))/2/pi)
\end{code}
The four spectra are unimodal and only the estimated spectrum can
be considered as a broad band spectrum.
For narrow band spectra, with one  mode, the
mean  frequency $f_0$ is close to the the mode of the spectrum.
Check it for your choice of spectrum.
\begin{code}
>> wspecplot(S);
>> 2*pi*f0
\end{code}



\textbf{ Remark.} The parameters $f_0$ and $\sigma$  are important
to perform a crude
analysis of the damage intensity of the load.
The mean frequency indicates a rate of
bigger cycles while $\sigma$ is the scaling factor for amplitudes.
The so-called moment method postulates that amplitudes of these cycles are
$\sigma\cdot R$, Rayleigh distributed (the pdf for a  Rayleigh variable is
$r\exp(-0.5r^2)$). This gives a very simple predictor of fatigue life time
$T^f$
$$
\hat T^f=\frac{1}{\gamma\,f_0E[(\sigma R)^\beta]}=
\frac{1}{\gamma\,f_0\sigma^\beta\,2^{\beta/2}\Gamma(\frac{\beta}{2}+1)}.
$$
As a matter of fact, one can prove that this predictor
is always conservative (gives
too short fatigue life than the one obtained by means of
rainflow method). Summarizing,
knowing $f_0$ and $\sigma$, the upper bound for the damage intensity can
be calculated. This is what the function \verb|spec2dplus| is computing.
If the spectrum is unimodal and particularly if it
is narrow banded, the method gives very accurate results.
Observe that in some special cases, when $\beta=1$ and as  $\beta$ goes to
infinity, the method is exact.
In addition,
this approach can be extended to non-Gaussian loads; then the crossing spectrum
$lc(u)$ has to be given. When the crossing spectrum $lc(u)$ is known,
one can compute conservative bounds for the
damage intensity by means of \verb|lc2dplus|. (Obviously for Gaussian loads,
 \verb|lc2dplus| gives the same results as \verb|spec2dplus|).

For the chosen spectrum, compute the conservative bound for the
predictor  of $T^f$, time to fatigue
failure. Use $\beta=3.0, 3.1, 3.2,\ldots, 5.0$ and
$\gamma=5\cdot10^{-9}$.
(Obviously, our examples are somewhat artificial since
the signals are not stresses and the parameters $\beta$ and $\gamma$
are not based on real tests.)
Computation of $\hat T^f$
\begin{code}
>> beta=3:0.1:5;
>> gam=5E-9;
>> dpl=gam*spec2dplus(S,beta);
>> Tfpl=(1./dpl)/3600/24  %in days
>> clf, plot(beta,Tfpl)
\end{code}


\section{Simulation of $X$ and the damage intensity}


In this section we shall simulate $X(t)$, extract rainflow cycles,
estimate damage intensity and derive an estimate of time to fatigue
failure $T^f$. If your spectrum is narrow banded, you should obtain
results close to the conservative bounds from the previous subsection.
For broad banded spectra the difference can be more relevant and
one may need to use other methods to derive estimates or $T^f$.

Suppose that one wishes to simulate $X(t)$ with approximatively
1000  non-negligible cycles. The duration time $T$ of the
signal is then $T=1000/f_0$. Compute (approximatively) the proportion
of fatigue lifetime consumed by $X(t)$, $0\le t\le T$,
in $\%$ ($100\%\,\,T/T^f$), for $\beta=4.22$ and $\gamma=5\cdot10^{-9}$.
Solution:
\begin{code}
>> T=1000/f0 % in seconds
>> gam=5E-9;
>> 100*T*gam*spec2dplus(S,4.22) % in percent
\end{code}


\textbf{Remark.} When one wishes to simulate the process $X(t)$ for
a given spectrum $S$, it may not be obvious how to choose an
appropriate sampling frequency. One possibility is to use the
highest angular frequency in the spectrum $\omega_{max}$, say. Then
a natural sampling frequency would be $f=\omega_{max}/2\pi$.
However, this is often a too low frequency to get correct values of
rainflow amplitudes. Simply, one would miss exact positions of local
maxima and minima in $X$, leading to  underestimation of rainflow
amplitudes. In many cases the choice of between 50 to 100 times
higher frequency than the mean frequency is giving good results. In
the following  example we will use ~$f=60f_0$, that gives 60000
observations for the chosen $T$ value.

We will simulate five sample paths of loads and compute the damage
intensity. The sampling interval is chosen to $dt=1/(60f_0)$. For
fast and accurate simulation, the spectrum is recomputed so that the
highest frequency $\omega_{max}=2\pi/dt$. Then $X(t)$ is simulated.
From the sample path, turning points are extracted and the rainflow
cycles are calculated. The damage intensity can then be computed,
and a prediction of the fatigue failure time $T^f$ obtained.
\begin{code}
>> max(S.w)/pi
>> dt=1./f0/60
>> S1=specinterp(S,dt);
>> [max(S1.w) pi/dt]
>> clf, wspecplot(S), hold on, wspecplot(S1,1,'r.')
>> clf, plot(beta,Tfpl)
>> hold on
>> for  i=1:5
>>   X=spec2sdat(S1,60000);
>>   tp=dat2tp(X);
>>   rfc=tp2rfc(tp);
>>   db=cc2dam(rfc,beta);
>>   plot(beta,(1000/f0)*(1./db)*(1/gam)/3600/24,'r.') % Tf in days
>> end
\end{code}


\section{Theoretical computation of damage intensity}

For more complicated spectra that are not unimodal, the moment method can be
too conservative. In order to illustrate this we choose the following spectrum
$$
S(\omega)=9\exp(-8(\omega-0.5)^2)
+2\exp(-2|w-2|^{1.4}).
$$
It can be created as follows
\begin{code}
>> S=createspec;
>> w=levels([0 4 253]);
>> S.S=2*exp(-2*abs(w-2).^1.4);
>> S.S=S.S+9*exp(-8*(w-0.5).^2);
>> S.w=w;
>> S.note=['S(w)=9*exp(-8(w-0.5)^2)+2*exp(-2|w-2|.^1.4)']
>> wspecplot(S)
\end{code}
Check the accuracy of the conservative bound for the damage intensity
by simulating 10 sample paths of the load.
\begin{code}
>> L=spec2mom(S,4);
>> f0=sqrt(L(2)/L(1))/2/pi
>> dpl=gam*spec2dplus(S,beta);
>> Tfpl=(1./dpl)/3600/24  %in days
>> dt=1./f0/60
>> S1=specinterp(S,dt);
>> figure(1)
>> clf, plot(beta,Tfpl)
>> hold on
>> for  i=1:10
>>   X=spec2sdat(S1,60000);
>>   tp=dat2tp(X);
>>   rfc=tp2rfc(tp);
>>   db=cc2dam(rfc,beta);
>>   plot(beta,(1000/f0)*(1./db)*(1/gam)/3600/24,'r.') % Tf in days
>> end
\end{code}
We turn now to alternative method of computation of the rainflow matrix
and damage intensity. The method consists of two steps. First, the
transition matrix from maximum to the following minimum
\verb|fMm|, say, in $X$, is computed using a generalized Rice's formula.
(Since numerical
integrations are needed, this step is relatively slow. The
accuracy parameter is called \verb|nit|$=0,\ldots,5$.
Higher \verb|nit| gives better approximation).
Next, the sequence of turning points is approximated by a MCTP. The methods
presented in Computer Exercise 2 then give the rainflow matrix.
\begin{code}
>> a=sqrt(L(3)/L(2))/2/pi
>> s=sqrt(L(1))
>> help spec2cmat
>> paramu=[-4.5*s 4.5*s 46];
>> nit=2;
>> figure(2), clf
>> [frfc fMm]=spec2cmat(S,[],'rfc',[],paramu,nit);
>> hold on, plot(rfc(:,2),rfc(:,1),'.')
>> clf, pdfplot(fMm)
>> dg=a*cmat2dam(paramu,frfc.f,beta);
>> figure(1), plot(beta,(1./dg)*(1/gam)/3600/24,'g') % Tf in days
\end{code}
For this broad banded spectrum the parameter \verb|nit|$=2$ is too low;
we propose to use \verb|nit|$=3$, but it can take 4 minutes to get
the results.
\begin{code}
>> figure(2), clf
>> [frfc1 fMm1]=spec2cmat(S,[],'rfc',[],paramu,3);
>> dg1=a*cmat2dam(paramu,frfc1.f,beta);
>> figure(1)
>> plot(beta,(1./dg1)*(1/gam)/3600/24,'k') % Tf in days
\end{code}
We turn now to some additional applications of the computed matrix
\verb|fMm|. (Those exercises are optional.)

First, the matrix can be used to simulate much longer and faster
sequences of turning points of $X$, than using \verb|spec2sdt|.
\begin{code}
>> help mctpsim
>> F{1,1}=fMm.f;
>> F{1,2}=fMm.f';
>> mctp=mctpsim(F,1000);
>> clf,plot(mctp(1:40))
\end{code}
Next, if the simulated sequence contains too many small amplitudes one can
derive the Markov matrix for the rainflow filtered sequence. It can be done as
follows: replace a few sub-diagonals in the \verb|frfc| matrix by zeros,
then invert  it to get the new transition matrix. Now you can
use it to simulate the filtered sequence of turning points.
\begin{code}
>> frfc1=zeros(size(frfc.f));
>> frfc1=triu(frfc.f,5);
>> fMm1=iter(frfc1,fMm.f,20,0.001);
>> clf,contour(fMm.f,20)
>> hold on,  contour(fMm1,20)
>> F{1,1}=fMm1;
>> F{1,2}=fMm1';
>> mctp1=mctpsim(F,1000);
>> clf, subplot(2,1,1)
>> plot(mctp(1:40))
>> subplot(2,1,2)
>> plot(mctp1(1:40))
\end{code}
